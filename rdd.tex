% Options for packages loaded elsewhere
\PassOptionsToPackage{unicode}{hyperref}
\PassOptionsToPackage{hyphens}{url}
\PassOptionsToPackage{dvipsnames,svgnames,x11names}{xcolor}
%
\documentclass[
  letterpaper,
  DIV=11,
  numbers=noendperiod]{scrreprt}

\usepackage{amsmath,amssymb}
\usepackage{iftex}
\ifPDFTeX
  \usepackage[T1]{fontenc}
  \usepackage[utf8]{inputenc}
  \usepackage{textcomp} % provide euro and other symbols
\else % if luatex or xetex
  \usepackage{unicode-math}
  \defaultfontfeatures{Scale=MatchLowercase}
  \defaultfontfeatures[\rmfamily]{Ligatures=TeX,Scale=1}
\fi
\usepackage{lmodern}
\ifPDFTeX\else  
    % xetex/luatex font selection
\fi
% Use upquote if available, for straight quotes in verbatim environments
\IfFileExists{upquote.sty}{\usepackage{upquote}}{}
\IfFileExists{microtype.sty}{% use microtype if available
  \usepackage[]{microtype}
  \UseMicrotypeSet[protrusion]{basicmath} % disable protrusion for tt fonts
}{}
\makeatletter
\@ifundefined{KOMAClassName}{% if non-KOMA class
  \IfFileExists{parskip.sty}{%
    \usepackage{parskip}
  }{% else
    \setlength{\parindent}{0pt}
    \setlength{\parskip}{6pt plus 2pt minus 1pt}}
}{% if KOMA class
  \KOMAoptions{parskip=half}}
\makeatother
\usepackage{xcolor}
\setlength{\emergencystretch}{3em} % prevent overfull lines
\setcounter{secnumdepth}{-\maxdimen} % remove section numbering
% Make \paragraph and \subparagraph free-standing
\ifx\paragraph\undefined\else
  \let\oldparagraph\paragraph
  \renewcommand{\paragraph}[1]{\oldparagraph{#1}\mbox{}}
\fi
\ifx\subparagraph\undefined\else
  \let\oldsubparagraph\subparagraph
  \renewcommand{\subparagraph}[1]{\oldsubparagraph{#1}\mbox{}}
\fi

\usepackage{color}
\usepackage{fancyvrb}
\newcommand{\VerbBar}{|}
\newcommand{\VERB}{\Verb[commandchars=\\\{\}]}
\DefineVerbatimEnvironment{Highlighting}{Verbatim}{commandchars=\\\{\}}
% Add ',fontsize=\small' for more characters per line
\usepackage{framed}
\definecolor{shadecolor}{RGB}{241,243,245}
\newenvironment{Shaded}{\begin{snugshade}}{\end{snugshade}}
\newcommand{\AlertTok}[1]{\textcolor[rgb]{0.68,0.00,0.00}{#1}}
\newcommand{\AnnotationTok}[1]{\textcolor[rgb]{0.37,0.37,0.37}{#1}}
\newcommand{\AttributeTok}[1]{\textcolor[rgb]{0.40,0.45,0.13}{#1}}
\newcommand{\BaseNTok}[1]{\textcolor[rgb]{0.68,0.00,0.00}{#1}}
\newcommand{\BuiltInTok}[1]{\textcolor[rgb]{0.00,0.23,0.31}{#1}}
\newcommand{\CharTok}[1]{\textcolor[rgb]{0.13,0.47,0.30}{#1}}
\newcommand{\CommentTok}[1]{\textcolor[rgb]{0.37,0.37,0.37}{#1}}
\newcommand{\CommentVarTok}[1]{\textcolor[rgb]{0.37,0.37,0.37}{\textit{#1}}}
\newcommand{\ConstantTok}[1]{\textcolor[rgb]{0.56,0.35,0.01}{#1}}
\newcommand{\ControlFlowTok}[1]{\textcolor[rgb]{0.00,0.23,0.31}{#1}}
\newcommand{\DataTypeTok}[1]{\textcolor[rgb]{0.68,0.00,0.00}{#1}}
\newcommand{\DecValTok}[1]{\textcolor[rgb]{0.68,0.00,0.00}{#1}}
\newcommand{\DocumentationTok}[1]{\textcolor[rgb]{0.37,0.37,0.37}{\textit{#1}}}
\newcommand{\ErrorTok}[1]{\textcolor[rgb]{0.68,0.00,0.00}{#1}}
\newcommand{\ExtensionTok}[1]{\textcolor[rgb]{0.00,0.23,0.31}{#1}}
\newcommand{\FloatTok}[1]{\textcolor[rgb]{0.68,0.00,0.00}{#1}}
\newcommand{\FunctionTok}[1]{\textcolor[rgb]{0.28,0.35,0.67}{#1}}
\newcommand{\ImportTok}[1]{\textcolor[rgb]{0.00,0.46,0.62}{#1}}
\newcommand{\InformationTok}[1]{\textcolor[rgb]{0.37,0.37,0.37}{#1}}
\newcommand{\KeywordTok}[1]{\textcolor[rgb]{0.00,0.23,0.31}{#1}}
\newcommand{\NormalTok}[1]{\textcolor[rgb]{0.00,0.23,0.31}{#1}}
\newcommand{\OperatorTok}[1]{\textcolor[rgb]{0.37,0.37,0.37}{#1}}
\newcommand{\OtherTok}[1]{\textcolor[rgb]{0.00,0.23,0.31}{#1}}
\newcommand{\PreprocessorTok}[1]{\textcolor[rgb]{0.68,0.00,0.00}{#1}}
\newcommand{\RegionMarkerTok}[1]{\textcolor[rgb]{0.00,0.23,0.31}{#1}}
\newcommand{\SpecialCharTok}[1]{\textcolor[rgb]{0.37,0.37,0.37}{#1}}
\newcommand{\SpecialStringTok}[1]{\textcolor[rgb]{0.13,0.47,0.30}{#1}}
\newcommand{\StringTok}[1]{\textcolor[rgb]{0.13,0.47,0.30}{#1}}
\newcommand{\VariableTok}[1]{\textcolor[rgb]{0.07,0.07,0.07}{#1}}
\newcommand{\VerbatimStringTok}[1]{\textcolor[rgb]{0.13,0.47,0.30}{#1}}
\newcommand{\WarningTok}[1]{\textcolor[rgb]{0.37,0.37,0.37}{\textit{#1}}}

\providecommand{\tightlist}{%
  \setlength{\itemsep}{0pt}\setlength{\parskip}{0pt}}\usepackage{longtable,booktabs,array}
\usepackage{calc} % for calculating minipage widths
% Correct order of tables after \paragraph or \subparagraph
\usepackage{etoolbox}
\makeatletter
\patchcmd\longtable{\par}{\if@noskipsec\mbox{}\fi\par}{}{}
\makeatother
% Allow footnotes in longtable head/foot
\IfFileExists{footnotehyper.sty}{\usepackage{footnotehyper}}{\usepackage{footnote}}
\makesavenoteenv{longtable}
\usepackage{graphicx}
\makeatletter
\def\maxwidth{\ifdim\Gin@nat@width>\linewidth\linewidth\else\Gin@nat@width\fi}
\def\maxheight{\ifdim\Gin@nat@height>\textheight\textheight\else\Gin@nat@height\fi}
\makeatother
% Scale images if necessary, so that they will not overflow the page
% margins by default, and it is still possible to overwrite the defaults
% using explicit options in \includegraphics[width, height, ...]{}
\setkeys{Gin}{width=\maxwidth,height=\maxheight,keepaspectratio}
% Set default figure placement to htbp
\makeatletter
\def\fps@figure{htbp}
\makeatother

\KOMAoption{captions}{tableheading}
\makeatletter
\makeatother
\makeatletter
\makeatother
\makeatletter
\@ifpackageloaded{caption}{}{\usepackage{caption}}
\AtBeginDocument{%
\ifdefined\contentsname
  \renewcommand*\contentsname{Table of contents}
\else
  \newcommand\contentsname{Table of contents}
\fi
\ifdefined\listfigurename
  \renewcommand*\listfigurename{List of Figures}
\else
  \newcommand\listfigurename{List of Figures}
\fi
\ifdefined\listtablename
  \renewcommand*\listtablename{List of Tables}
\else
  \newcommand\listtablename{List of Tables}
\fi
\ifdefined\figurename
  \renewcommand*\figurename{Figure}
\else
  \newcommand\figurename{Figure}
\fi
\ifdefined\tablename
  \renewcommand*\tablename{Table}
\else
  \newcommand\tablename{Table}
\fi
}
\@ifpackageloaded{float}{}{\usepackage{float}}
\floatstyle{ruled}
\@ifundefined{c@chapter}{\newfloat{codelisting}{h}{lop}}{\newfloat{codelisting}{h}{lop}[chapter]}
\floatname{codelisting}{Listing}
\newcommand*\listoflistings{\listof{codelisting}{List of Listings}}
\makeatother
\makeatletter
\@ifpackageloaded{caption}{}{\usepackage{caption}}
\@ifpackageloaded{subcaption}{}{\usepackage{subcaption}}
\makeatother
\makeatletter
\@ifpackageloaded{tcolorbox}{}{\usepackage[skins,breakable]{tcolorbox}}
\makeatother
\makeatletter
\@ifundefined{shadecolor}{\definecolor{shadecolor}{rgb}{.97, .97, .97}}
\makeatother
\makeatletter
\makeatother
\makeatletter
\makeatother
\ifLuaTeX
  \usepackage{selnolig}  % disable illegal ligatures
\fi
\IfFileExists{bookmark.sty}{\usepackage{bookmark}}{\usepackage{hyperref}}
\IfFileExists{xurl.sty}{\usepackage{xurl}}{} % add URL line breaks if available
\urlstyle{same} % disable monospaced font for URLs
\hypersetup{
  colorlinks=true,
  linkcolor={blue},
  filecolor={Maroon},
  citecolor={Blue},
  urlcolor={Blue},
  pdfcreator={LaTeX via pandoc}}

\author{}
\date{}

\begin{document}
\ifdefined\Shaded\renewenvironment{Shaded}{\begin{tcolorbox}[breakable, boxrule=0pt, interior hidden, enhanced, borderline west={3pt}{0pt}{shadecolor}, frame hidden, sharp corners]}{\end{tcolorbox}}\fi

\hypertarget{regression-discontinuity-design}{%
\chapter{Regression Discontinuity
Design}\label{regression-discontinuity-design}}

\begin{Shaded}
\begin{Highlighting}[]
\NormalTok{knitr}\SpecialCharTok{::}\NormalTok{opts\_chunk}\SpecialCharTok{$}\FunctionTok{set}\NormalTok{(}
    \AttributeTok{echo =} \ConstantTok{TRUE}\NormalTok{,}
    \AttributeTok{message =} \ConstantTok{FALSE}\NormalTok{,}
    \AttributeTok{warning =} \ConstantTok{FALSE}\NormalTok{,}
    \AttributeTok{comment =} \ConstantTok{NA}
\NormalTok{)}

\FunctionTok{rm}\NormalTok{(}\AttributeTok{list =} \FunctionTok{ls}\NormalTok{())}

\FunctionTok{library}\NormalTok{(readr)}
\NormalTok{educ }\OtherTok{\textless{}{-}} \FunctionTok{read\_csv}\NormalTok{(}\StringTok{"islamic\_women.csv"}\NormalTok{)}
\end{Highlighting}
\end{Shaded}

\hypertarget{lecture-slides-data}{%
\subsection{Lecture Slides \& Data}\label{lecture-slides-data}}

\hfill\break
\hfill\break

\begin{Shaded}
\begin{Highlighting}[]
\CommentTok{\#install.packages("downloadthis")}
\FunctionTok{library}\NormalTok{(downloadthis)}

\FunctionTok{download\_link}\NormalTok{(}
  \AttributeTok{link =} \StringTok{"https://bayreuth{-}politics.github.io/CI22/data/islamic\_women.csv"}\NormalTok{,}
  \AttributeTok{output\_name =} \StringTok{"educ"}\NormalTok{,}
  \AttributeTok{output\_extension =} \StringTok{".rdata"}\NormalTok{,}
  \AttributeTok{button\_label =} \StringTok{"Lab 8 Data"}\NormalTok{,}
  \AttributeTok{button\_type =} \StringTok{"success"}\NormalTok{,}
  \AttributeTok{has\_icon =} \ConstantTok{TRUE}\NormalTok{,}
  \AttributeTok{self\_contained =} \ConstantTok{TRUE}
\NormalTok{)}

\CommentTok{\#download\_link(}
\CommentTok{\#  link = "https://github.com/dpir{-}ci/CI22/raw/gh{-}pages/docs/lectures/lecture7.pdf",}
\CommentTok{\#  output\_name = "week7",}
\CommentTok{\#  output\_extension = ".pdf",}
\CommentTok{\#  button\_label = "Lecture Slides",}
\CommentTok{\#  button\_type = "default",}
\CommentTok{\#  has\_icon = FALSE,}
\CommentTok{\#  self\_contained = FALSE}
\CommentTok{\#)}
\end{Highlighting}
\end{Shaded}

\hfill\break
\hfill\break

\begin{center}\rule{0.5\linewidth}{0.5pt}\end{center}

\hypertarget{recap}{%
\chapter{Recap}\label{recap}}

\hypertarget{sharp-regression-discontinuity}{%
\subsection{Sharp Regression
Discontinuity}\label{sharp-regression-discontinuity}}

In RDDs we exploit that treatment assignment is determined by a known
\textbf{assignment rule} that determines whether units are assigned to
the treatment. In RDD, all units in the study receive a score, usually
called \emph{running variable}, \emph{forcing variable} or \emph{index},
and treatment is assigned to those units whose score is above a known
cut-off (Cattaneo et al 2019).

In RDD we have three components:

\begin{enumerate}
\def\labelenumi{\arabic{enumi}.}
\tightlist
\item
  The score
\item
  The cutoff
\item
  The treatment.
\end{enumerate}

In the Sharp Regression Discontinuity Design, the score determines -
deterministically - whether the unit is being assigned to treatment or
to the control condition. However, we again face the fundamental problem
of causal inference because we can only observe the untreated outcome
for those units below the cutoff (control) and the treated outcome for
those above the cut-off (treated). However, imposing the assumption of
comparability between units with very similar values of the score but on
opposite sides of cut-off enables us to calculate the treatment effect
of the intervention.

Given the continuity assumption, we would expect that observations in a
small neighbourhood around the cut-off will have very similar potential
outcomes. Thus, this would justify using observations just below the
cut-off as a reasonable proxy of what would the average outcome for
those units just above the cut-off would have had if they had received
the control condition instead of the treatment (i.e.~counterfactual).

The main goal in RD is to adequately perform an extrapolation of the
average outcome of treated and untreated units at the cutoff.

\hypertarget{estimation}{%
\subsection{Estimation}\label{estimation}}

There are different ways that we could estimate the causal effect from
RDD:

\begin{itemize}
\tightlist
\item
  Linear
\item
  Linear with different slopes
\item
  High-order polynomial (or non-linear)
\item
  Non-parametric
\end{itemize}

\hypertarget{choice-of-kernel-function-and-polynomial-order}{%
\subsection{Choice of Kernel function and Polynomial
Order}\label{choice-of-kernel-function-and-polynomial-order}}

Whenever we conduct RDD we choose the following:

\begin{enumerate}
\def\labelenumi{\arabic{enumi}.}
\tightlist
\item
  Polynomial order \(p\) function
\item
  Kernel function \(K(\dot)\), which determines how observations within
  a bandwidth would be weighted, and there are different options:
  ``Uniform'', ``Triangular'', and ``Epanechnikov''.
\item
  Bandwidth size: There is a trade-off between bias and efficiency. The
  closer you get to the cut-off, the less bias in the estimator, but
  more variance as there are fewer observations.
\item
  We conduct weighted least squares and obtain the intercept of the
  chosen polynomial order above and below the cut-off.
\item
  Calculate the sharp point estimate by subtracting the intercepts of
  these polynomials:
  \(\hat{\tau_{SRD}} = \alpha_{above} - \alpha_{below}\)
\end{enumerate}

\hfill\break

\hypertarget{falsification-tests}{%
\subsection{Falsification tests:}\label{falsification-tests}}

Whenever we conduct RD we choose the following:

We learned that we can conduct multiple sensitivity and falsification
tests:

\begin{enumerate}
\def\labelenumi{\arabic{enumi}.}
\tightlist
\item
  Sensitivity: To check that the results of our estimation are robust to
  different specifications
\item
  Continuity: To check whether covariates do not jump at the threshold.
\item
  Sorting: To check that units do not sort around the threshold.
\item
  Placebo cut-offs: To check that outcomes do not change abruptly at an
  arbitrary threshold (to test the continuity assumption)
\end{enumerate}

\begin{center}\rule{0.5\linewidth}{0.5pt}\end{center}

\textbf{Before starting this seminar}

\begin{enumerate}
\def\labelenumi{\arabic{enumi}.}
\item
  Create a folder called ``lab8''
\item
  Download the data (you can use the button or the one at the top, or
  read csv files directly from github):
\item
  Open an R script (or Markdown file) and save it in our ``lab8''
  folder.
\item
  Set your working directory using the setwd() function or by clicking
  on ``More``. For example \emph{setwd(``\textasciitilde/Desktop/Causal
  Inference/2022/Lab8'')}
\item
  Let's install an load packages that we will be using in this lab:
\end{enumerate}

\begin{Shaded}
\begin{Highlighting}[]
\FunctionTok{library}\NormalTok{(stargazer) }\CommentTok{\# generate formated regression tables }
\FunctionTok{library}\NormalTok{(texreg) }\CommentTok{\# generate formatted regression tables}
\FunctionTok{library}\NormalTok{(tidyverse) }\CommentTok{\# to conduct some tidy operations}
\FunctionTok{library}\NormalTok{(ggplot2)}
\CommentTok{\#install.packages(c(rdrobust, rddensity))}
\FunctionTok{library}\NormalTok{(tidyverse)}
\CommentTok{\#install.packages("rdd")}
\FunctionTok{library}\NormalTok{(rdd) }\CommentTok{\# to conduct McCrary Sorting Test }
\FunctionTok{library}\NormalTok{(rdrobust) }\CommentTok{\# to conduct non{-}parametric rd estimates }
\FunctionTok{library}\NormalTok{(rddensity) }\CommentTok{\# to conduct a density test and density plots}
\end{Highlighting}
\end{Shaded}

\begin{center}\rule{0.5\linewidth}{0.5pt}\end{center}

\hypertarget{seminar-overview}{%
\chapter{Seminar Overview}\label{seminar-overview}}

In this \textbf{seminar}, we will cover the following topics:\\
1. Conduct regression discontinuity using global polynomial estimation
using \texttt{lm()} function\\
2. Calculate LATE using non-parametric estimations using
\texttt{rdrobust()}.\\
3. Conduct various robustness/falsification tests such as balance test,
placebo outcome, density test, and falsification test.

\begin{center}\rule{0.5\linewidth}{0.5pt}\end{center}

\hypertarget{islamic-rule-and-the-empowerment-of-the-poor-and-pious---meyerson-2014}{%
\section{Islamic Rule and the Empowerment of the Poor and Pious -
Meyerson
(2014)}\label{islamic-rule-and-the-empowerment-of-the-poor-and-pious---meyerson-2014}}

In this
\href{https://onlinelibrary.wiley.com/doi/abs/10.3982/ECTA9878}{paper},
Meyerson is interested in the effects of Islamic parties' control of
local governments on women's rights. He focuses on the educational
attainment of young women. Meyerson conducts a Sharp RD design, based on
close elections in Turkey. The challenge here is to compare
municipalities where the support for Islamic parties is high and win the
election, versus those that elected a secular mayor.

You would expect that municipalities controlled by Islamic parties would
systematically differ from those that are controlled by a secular mayor.
Particularly, if religious conservatism affects the educational outcomes
of women. However, we can use RDD to isolate the treatment effect of
interest from all systematic differences between treated and untreated
units.

We can compare municipalities the Islamic party barely won the election
versus municipalities where the Islamic party barely lost. This reveals
the causal (local) effect of Islamic party control on women's
educational attainment a few years later. One crucial condition to meet
in this setup is that parties cannot systematically manipulate the vote
share they obtain.

The data used in this study is from the 2014 mayoral election in Turkey.
The unit of analysis is the municipality, and the running variable is
the margin of victory. The outcome of interest is the educational
attainment of women who attended high school during 1994-2000,
calculated as a percentage of the cohort of women aged 15 to 20 in 2000
who had completed high school by 2000.

We will be using the following variables:

\begin{longtable}[]{@{}
  >{\raggedright\arraybackslash}p{(\columnwidth - 2\tabcolsep) * \real{0.1667}}
  >{\raggedright\arraybackslash}p{(\columnwidth - 2\tabcolsep) * \real{0.8333}}@{}}
\toprule\noalign{}
\begin{minipage}[b]{\linewidth}\raggedright
Variable
\end{minipage} & \begin{minipage}[b]{\linewidth}\raggedright
Description
\end{minipage} \\
\midrule\noalign{}
\endhead
\bottomrule\noalign{}
\endlastfoot
\texttt{margin} & This variable represents the margin of victory of
Islamic parties in the 1994 election. A positive margin means that an
Islamic party won. \\
\texttt{school\_men} & secondary school completion rate for men aged
between 15 and 20 \\
\texttt{school\_women} & the secondary school completion rate for women
aged 15-20 \\
\texttt{log\_pop} & log of the municipality population in 1994 \\
\texttt{sex\_ratio} & gender ratio of the municipality in 1994 \\
\texttt{log\_area} & log of the municipality area in 1994 \\
\end{longtable}

\hfill\break

Now let's load the data. There are two ways to do this:

You can load the dataset from your laptop using the \texttt{read.csv()}
function. Here the dataset is called \texttt{educ} - but feel free to
give it a different name if you prefer.

\begin{Shaded}
\begin{Highlighting}[]
\CommentTok{\# Set your working directory}
\CommentTok{\#setwd("\textasciitilde{}/Desktop/Causal Inference/2022/Lab8")}
\CommentTok{\# }
\FunctionTok{library}\NormalTok{(readr)}
\CommentTok{\#educ \textless{}{-} read.csv("\textasciitilde{}/islamic\_women.csv")}

\FunctionTok{head}\NormalTok{(educ)}
\end{Highlighting}
\end{Shaded}

Let's start by visualising the data. We will use \texttt{plot()}
function to do this. This is the simplest scatter plot that you can come
up with.

\textbf{Exercise 1: Generate a plot using the \texttt{plot(X,Y)}
function. Replace X with \texttt{educ\$margin} and Y with
\texttt{educ\$school\_women}.}

\hfill\break

\emph{Reveal Answer}

\begin{Shaded}
\begin{Highlighting}[]
\FunctionTok{plot}\NormalTok{(educ}\SpecialCharTok{$}\NormalTok{margin, educ}\SpecialCharTok{$}\NormalTok{school\_women)}
\end{Highlighting}
\end{Shaded}

This is a very simple plot that shows the raw relationship between the
margin of victory and the outcome variable. However, it conveys some
important information. For example, the margin of victory is clustered
around -0.5 and roughly 0.3. Also, the outcome variable, school
attainment, usually goes from 0 to 40\%. Now let's generate a slightly
fancier plot.

\hfill\break

\textbf{Exercise 2: Generate a scatter plot using \texttt{ggplot()}
function. Use the functions below to add some additional features into
this plot.}

\begin{Shaded}
\begin{Highlighting}[]
\FunctionTok{ggplot}\NormalTok{(}\FunctionTok{aes}\NormalTok{(}\AttributeTok{x =}\NormalTok{ running variable, }\AttributeTok{y =}\NormalTok{ outcome, }\AttributeTok{colour =}\NormalTok{outcome), }\AttributeTok{data =}\NormalTok{ data) }\SpecialCharTok{+}
  \CommentTok{\# Make points small and semi{-}transparent since there are lots of them}
  \FunctionTok{geom\_point}\NormalTok{(}\AttributeTok{size =} \FloatTok{0.5}\NormalTok{, }\AttributeTok{alpha =} \FloatTok{0.5}\NormalTok{, }\AttributeTok{position =} \FunctionTok{position\_jitter}\NormalTok{(}\AttributeTok{width =} \DecValTok{0}\NormalTok{, }\AttributeTok{height =} \FloatTok{0.25}\NormalTok{, }\AttributeTok{seed =} \DecValTok{1234}\NormalTok{)) }\SpecialCharTok{+} 
  \CommentTok{\# Add vertical line}
  \FunctionTok{geom\_vline}\NormalTok{(}\AttributeTok{xintercept =} \DecValTok{0}\NormalTok{) }\SpecialCharTok{+} 
  \CommentTok{\# Add labels}
  \FunctionTok{labs}\NormalTok{(}\AttributeTok{x =} \StringTok{"Label X"}\NormalTok{, }\AttributeTok{y =} \StringTok{"Label Y"}\NormalTok{) }\SpecialCharTok{+} 
  \CommentTok{\# Turn off the color legend, since it\textquotesingle{}s redundant}
  \FunctionTok{guides}\NormalTok{(}\AttributeTok{color =} \ConstantTok{FALSE}\NormalTok{)}
\end{Highlighting}
\end{Shaded}

\hfill\break

\emph{Reveal Answer}

\begin{Shaded}
\begin{Highlighting}[]
\CommentTok{\# Let\textquotesingle{}s check if this is a sharp RD. }
\FunctionTok{ggplot}\NormalTok{(educ, }\FunctionTok{aes}\NormalTok{(}\AttributeTok{x =}\NormalTok{ margin, }\AttributeTok{y =}\NormalTok{ school\_women, }\AttributeTok{colour =}\NormalTok{school\_women)) }\SpecialCharTok{+}
  \CommentTok{\# Make points small and semi{-}transparent since there are lots of them}
  \FunctionTok{geom\_point}\NormalTok{(}\AttributeTok{size =} \FloatTok{0.5}\NormalTok{, }\AttributeTok{alpha =} \FloatTok{0.5}\NormalTok{, }
             \AttributeTok{position =} \FunctionTok{position\_jitter}\NormalTok{(}\AttributeTok{width =} \DecValTok{0}\NormalTok{, }\AttributeTok{height =} \FloatTok{0.25}\NormalTok{, }\AttributeTok{seed =} \DecValTok{1234}\NormalTok{)) }\SpecialCharTok{+} 
  \CommentTok{\# Add vertical line}
  \FunctionTok{geom\_vline}\NormalTok{(}\AttributeTok{xintercept =} \DecValTok{0}\NormalTok{) }\SpecialCharTok{+} 
  \CommentTok{\# Add labels}
  \FunctionTok{labs}\NormalTok{(}\AttributeTok{x =} \StringTok{"Vote share"}\NormalTok{, }\AttributeTok{y =} \StringTok{"Womens\textquotesingle{} Educational Attainment (Proportion)"}\NormalTok{) }\SpecialCharTok{+} 
  \CommentTok{\# Turn off the color legend, since it\textquotesingle{}s redundant}
  \FunctionTok{guides}\NormalTok{(}\AttributeTok{color =} \ConstantTok{FALSE}\NormalTok{)}
\end{Highlighting}
\end{Shaded}

We can now see more clearly that most municipalities elected a secular
mayor. Recall that we are using the margin of victory for the Islamic
parties - a positive margin of victory is positive means that the
Islamic parties won the election in that municipality.

\hfill\break

\textbf{Exercise 3: Let's generate a dummy variable that is equal to
``Treated'' if the margin of victory \emph{is equal or greater than
zero} and ``Untreated'' otherwise. You can use the \texttt{mutate()}
function and the \texttt{ifelse()} functions to do this. Remember to use
the pipeline operator to store the variable into your existing data
frame. See below the syntax for more information.}

\begin{longtable}[]{@{}
  >{\raggedright\arraybackslash}p{(\columnwidth - 2\tabcolsep) * \real{0.5600}}
  >{\raggedright\arraybackslash}p{(\columnwidth - 2\tabcolsep) * \real{0.4400}}@{}}
\toprule\noalign{}
\begin{minipage}[b]{\linewidth}\raggedright
Function/argument
\end{minipage} & \begin{minipage}[b]{\linewidth}\raggedright
Description
\end{minipage} \\
\midrule\noalign{}
\endhead
\bottomrule\noalign{}
\endlastfoot
\texttt{data\ \textless{}-\ data\ \%\textgreater{}\%} & Pipeline
operator to assign the new operation into a new data or existing data
frame \\
\texttt{mutate(new\ variable\ =\ ifelse(variable\ \textgreater{}=\ "condition",\ "Treated",\ "Untreated")}
& If the condition is met, the new variable takes value equal to
``Treated'' and ``Untreated'' otherwise \\
\end{longtable}

\begin{Shaded}
\begin{Highlighting}[]
\NormalTok{data }\OtherTok{\textless{}{-}}\NormalTok{ data }\SpecialCharTok{\%\textgreater{}\%} 
  \FunctionTok{mutate}\NormalTok{(}\AttributeTok{newvariable =} \FunctionTok{ifelse}\NormalTok{(variable }\SpecialCharTok{\textgreater{}=} \DecValTok{0}\NormalTok{ , }\StringTok{"Treated"}\NormalTok{, }\StringTok{"Untreated"}\NormalTok{))}
\end{Highlighting}
\end{Shaded}

\hfill\break

\emph{Reveal Answer}

\begin{Shaded}
\begin{Highlighting}[]
\NormalTok{educ }\OtherTok{\textless{}{-}}\NormalTok{ educ }\SpecialCharTok{\%\textgreater{}\%} 
  \FunctionTok{mutate}\NormalTok{(}\AttributeTok{treat =} \FunctionTok{ifelse}\NormalTok{(margin }\SpecialCharTok{\textgreater{}=} \DecValTok{0}\NormalTok{ , }\StringTok{"Treated"}\NormalTok{, }\StringTok{"Untreated"}\NormalTok{))}
\end{Highlighting}
\end{Shaded}

Now that we have created our treatment condition variable, let's
generate an additional plot that conveys information on the distribution
of the running variable for both treated and untreated municipalities.

\hfill\break

\textbf{Exercise 4: Generate a plot looking at the distribution margin
of victory variable. Using the \texttt{ggplot()} function, set the
\texttt{x} argument equal to the margin of victory variable. Set
\texttt{fill} equal the new treatment variable \texttt{treat}. There is
no need to add the \texttt{y} argument given that we expect to generate
a histogram that will give us the number of observations for each value
in the margin of victory variable. Add the \texttt{geom\_histogram()}
function. Set the argument \texttt{binwidth} in this function equal to
0.01 and set \texttt{colour} argument equal to ``dark''. Let's add the
\texttt{geom\_vline()} function and add a vertical line by setting the
\texttt{xintercept} argument equal to zero. Add the \texttt{labs()}
function and set \texttt{x} label equal to ``Margin of victory'' and the
\texttt{y} label equal to ``count'', and the \texttt{fill} argument
equal to ``Treatment Status''.}

\hfill\break

\emph{Reveal Answer}

\begin{Shaded}
\begin{Highlighting}[]
\FunctionTok{ggplot}\NormalTok{(educ, }\FunctionTok{aes}\NormalTok{(}\AttributeTok{x =}\NormalTok{ margin, }\AttributeTok{fill =}\NormalTok{ treat)) }\SpecialCharTok{+}
  \FunctionTok{geom\_histogram}\NormalTok{(}\AttributeTok{binwidth =} \FloatTok{0.01}\NormalTok{, }\AttributeTok{color =} \StringTok{"white"}\NormalTok{) }\SpecialCharTok{+} 
  \FunctionTok{geom\_vline}\NormalTok{(}\AttributeTok{xintercept =} \DecValTok{0}\NormalTok{) }\SpecialCharTok{+} 
  \FunctionTok{labs}\NormalTok{(}\AttributeTok{x =} \StringTok{"Margin of Victory"}\NormalTok{, }\AttributeTok{y =} \StringTok{"Count"}\NormalTok{, }\AttributeTok{fill =} \StringTok{"Treatment Status"}\NormalTok{)}
\end{Highlighting}
\end{Shaded}

\hfill\break

In this plot, we can see the distribution of the running variable
(margin of victory). Again, we can see that in the majority of the
municipalities a secular mayor was elected.

\hfill\break

\hypertarget{global-parametric-estimation}{%
\section{Global Parametric
Estimation}\label{global-parametric-estimation}}

As we discussed in the lecture, there are different ways to estimate the
causal effect using RD. These different approaches differ in the range
of observations they include as well as how they estimate the average
outcome for those units just above the cut-off and below the cut-off.
One way to estimate the effect of the intervention in RDD is using OLS,
but only using the running variable as the main predictor. In this case,
we use the running variable measured as the distance from the cut-off.
Stated formally: \(\tilde{X_l} = X - c\).

In this case, the regression in the left-hand side would be equal to:

\[Y= \alpha_l + \tilde{X} + \epsilon\] Whereas the regression above the
cut-off is equal to:

\[Y= \alpha_r + \tilde{X} + \epsilon\] It's important to point out that
for all estimations the treatment effect is equal to the differences of
the intercepts of the regressions above and below the cut-off.

\[\tau = \alpha_r - \alpha_l\]

Let's do this manually. To do so, subset the data for those observations
above and below the threshold. Then, regress the outcome on the
\textbf{running variable}. Finally, subtract the intercepts from each
regression. Let's do that in the following exercise.

\textbf{Exercise 5: Run a regression using only \texttt{margin} variable
as the predictor. Set your outcome variable equal to
\texttt{school\_women} variable. Set the \texttt{data} argument equal to
\texttt{educ} (unless you called your data frame differently). Add the
\texttt{subset} function inside of the \texttt{lm()} function and set it
equal to \texttt{margin\ \textgreater{}=\ 0} for the regression above
the cut-off and \texttt{margin\ \textless{}\ 0} for the regression below
the cut-off. Use the \texttt{summary()} function to report your results,
but also store the output of this function into an object. You can
retrieve the intercept from the object that you stored the output from
the \texttt{summary()} function this way:}

\begin{Shaded}
\begin{Highlighting}[]
\NormalTok{object }\OtherTok{\textless{}{-}} \FunctionTok{summary}\NormalTok{(}\FunctionTok{lm}\NormalTok{(outcome }\SpecialCharTok{\textasciitilde{}}\NormalTok{ variable, }\AttributeTok{data =}\NormalTok{ data, }\AttributeTok{subset =}\NormalTok{ variable }\SpecialCharTok{\textgreater{}=}\NormalTok{ condition))}

\CommentTok{\# intercept}
\NormalTok{object}\SpecialCharTok{$}\NormalTok{coefficient[}\DecValTok{1}\NormalTok{]}
\end{Highlighting}
\end{Shaded}

\hfill\break

\emph{Reveal Answer}

\begin{Shaded}
\begin{Highlighting}[]
\CommentTok{\# above}
\NormalTok{above }\OtherTok{\textless{}{-}} \FunctionTok{summary}\NormalTok{(}\FunctionTok{lm}\NormalTok{(school\_women}\SpecialCharTok{\textasciitilde{}}\NormalTok{margin, }\AttributeTok{data =}\NormalTok{ educ, }\AttributeTok{subset =}\NormalTok{ margin }\SpecialCharTok{\textgreater{}=} \DecValTok{0}\NormalTok{))}

\NormalTok{above}\SpecialCharTok{$}\NormalTok{coefficients[}\DecValTok{1}\NormalTok{]}

\CommentTok{\# above}
\NormalTok{below }\OtherTok{\textless{}{-}} \FunctionTok{summary}\NormalTok{(}\FunctionTok{lm}\NormalTok{(school\_women}\SpecialCharTok{\textasciitilde{}}\NormalTok{margin, }\AttributeTok{data =}\NormalTok{ educ, }\AttributeTok{subset =}\NormalTok{ margin }\SpecialCharTok{\textless{}} \DecValTok{0}\NormalTok{))}

\NormalTok{below}\SpecialCharTok{$}\NormalTok{coefficients[}\DecValTok{1}\NormalTok{]}

\CommentTok{\# 0.156453 {-} 0.162002}
\NormalTok{tau\_rd }\OtherTok{=}\NormalTok{ above}\SpecialCharTok{$}\NormalTok{coefficients[}\DecValTok{1}\NormalTok{] }\SpecialCharTok{{-}}\NormalTok{ below}\SpecialCharTok{$}\NormalTok{coefficients[}\DecValTok{1}\NormalTok{]}
\NormalTok{tau\_rd}
\end{Highlighting}
\end{Shaded}

Based on this approach the intercepts of the two regressions yield the
estimated value of the average outcome at the cut-off point for the
treated and untreated units. This difference of intercepts is the
estimated effect of an Islamic party being in power (at the municipal
level) - it suggests there is a 0.5 percentage decrease in women's
educational attainment.

\hfill\break

A more direct way of estimating the treatment effect is to run a pooled
regression on both sides of the cut-off point, using the following
specification: \(Y = \alpha + \tau D + \beta \tilde{X} + \epsilon\)

Where \(\tau\) is the coefficient of interest. Here again LATE is the
difference between the two intercepts: \(\tau = \alpha_r -\alpha_l\).
When \(D\) switches off and we are also controlling the different values
of the forcing variable, \(\tilde{X}\), we get the slope of the
regression below the threshold. Conversely, for units above the cut-off,
\(D\) switches on, and we control for different values of the forcing
variable, we get the slope of the regression above the cut-off. The
estimated effect of the treatment at \(\tilde{X}\) then provides the
treatment effect (\(\tau\)).

Note that you are constraining the slope of the regression lines to the
same on both sides of the cut-off. (\(\beta_l = \beta_r\)) This might
not be consistent if the data structure varies and the single slope
fails to appropriately approximating each side to the cutoff.

\textbf{Exercise 6: Calculate the effect of the intervention using a
regression model including the \texttt{margin} and \texttt{treat}
variables. Use the\texttt{lm()} function to conduct this analysis. Store
this regression into an object and call it \texttt{global2}. Use the
\texttt{summary()} to inspect your results. Interpret the coefficient of
interest.}

\hfill\break

\emph{Reveal Answer}

\begin{Shaded}
\begin{Highlighting}[]
\CommentTok{\# a more direct way is to run a pooled regression on both sides of the cut{-}off (constraining the slopes)}

\NormalTok{global2 }\OtherTok{\textless{}{-}} \FunctionTok{lm}\NormalTok{(school\_women}\SpecialCharTok{\textasciitilde{}}\NormalTok{treat}\SpecialCharTok{+}\NormalTok{margin, }\AttributeTok{data =}\NormalTok{ educ)}
\FunctionTok{summary}\NormalTok{(global2)}
\end{Highlighting}
\end{Shaded}

Using this estimation approach, we obtain that, on average, the effect
of a municipality in control of an Islamic party leads to a 1.7\%-point
increase in women's educational attainment.

\hfill\break

We can also allow the regression function to differ on both sides of the
cut-off by including interaction terms between \(D\) and \(\tilde{X}\).
This would be as follows:

\[Y = \alpha_l + \tau D + \beta_0\tilde{X} + \beta_1 (D \times\tilde{X})+ \epsilon\]

Let's do that in the following exercise.

\textbf{Exercise 7: Calculate the effect of the intervention using a
regression model including the \texttt{margin} and \texttt{treat}
variables and the interaction between these two variables. Use
the\texttt{lm()} function to conduct this analysis. Store this
regression into an object and call it \texttt{global3}. Use the
\texttt{summary()} to inspect your results. Interpret the coefficient of
interest.}

\hfill\break

\emph{Reveal Answer}

\begin{Shaded}
\begin{Highlighting}[]
\CommentTok{\# a more direct way is to run a pooled regression on both sides of the cut{-}off (constraining the slopes)}

\NormalTok{global3 }\OtherTok{\textless{}{-}} \FunctionTok{lm}\NormalTok{(school\_women}\SpecialCharTok{\textasciitilde{}}\NormalTok{treat}\SpecialCharTok{+}\NormalTok{margin }\SpecialCharTok{+}\NormalTok{ treat}\SpecialCharTok{*}\NormalTok{margin, }\AttributeTok{data =}\NormalTok{ educ)}
\FunctionTok{summary}\NormalTok{(global3)}
\end{Highlighting}
\end{Shaded}

Here, we find that the coefficient of interest is positive
(\texttt{0.005}) yet insiginifcant. This means that, based on this
model, the effect of an Islamic party in control of the municipal
government does not lead to a change in women's educational attainment.

\hfill\break

Global parametric models have a severe shortcoming - they rely upon
observations that are far away from the cut-off. Indeed, the evidence
against using a global polynomial approach is quite substantial.
According to Cattaneo et al (2019), this estimation technique does not
provide accurate point estimators and inference procedures with good
statistical properties.

\textbf{Exercise 7 (no coding required): Think about how the global
polynomial approach weights each observation when it calculates the
coefficient of interest?}

\hfill\break

\emph{Reveal Answer}

\hfill\break

OLS will estimate \(\tau\) based on all observations across the score.
This means that the observations' very far from the cut-off weight is
equal to that of very close ones'. In the worst-case scenario, if the
observations are clustered far from the cut-off, the estimation of
\(\tau\) would be heavily influenced by those values rather than those
close.

\hfill\break

We can also use high-order polynomial to retrieve LATE. However, the
evidence against using high-order polynomial seems to be quite robust
\href{https://www.tandfonline.com/doi/abs/10.1080/07350015.2017.1366909}{see
here for a discussion on high-order polynomials)}. In short, the issues
with using high-order polynomials is that they leads to noisy estimates,
they are sensitivity to the degree of the polynomial, and they have poor
coverage of confidence intervals.

\textbf{Exercise 8: Well - let's give it a try nonetheless. Conduct a
third-order polynomial regression function. Include in this model the
\texttt{treat} and the \texttt{margin} variables. Also, add the
\texttt{margin} variable raised at the power of 2 and then at the power
of 3. We also need to include the \texttt{I()} function or insulate
function for the \texttt{margin} variable that is raised at the power of
2 and 3. The \texttt{I()} function insulates whatever is inside this
function. It creates a new predictor that is the product of the margin
variable by itself two and three times. Store this output in an object
and call that object \texttt{global4}. Then, use the \texttt{summary()}
to check the results of this specification. Interpret the results. }

\hfill\break

\emph{Reveal Answer}

\begin{Shaded}
\begin{Highlighting}[]
\NormalTok{global4 }\OtherTok{\textless{}{-}} \FunctionTok{lm}\NormalTok{(school\_women}\SpecialCharTok{\textasciitilde{}}\NormalTok{ treat }\SpecialCharTok{+}\NormalTok{ margin}\SpecialCharTok{+}\FunctionTok{I}\NormalTok{(margin}\SpecialCharTok{\^{}}\DecValTok{2}\NormalTok{)}\SpecialCharTok{+} \FunctionTok{I}\NormalTok{(margin}\SpecialCharTok{\^{}}\DecValTok{3}\NormalTok{) , }\AttributeTok{data =}\NormalTok{ educ)}

\FunctionTok{summary}\NormalTok{(global4)}
\end{Highlighting}
\end{Shaded}

Using a high-order polynomial function, we find that womens' educational
attainment decreases, on average, by roughly \texttt{2.1} per cent when
an Islamic party is in power - yet the result is insignificant.

\hfill\break

You might have reason to prefer parametric approaches - or deem them
more appropriate in some cases. Then, you should not resort to a global
model. It's more appropriate to only use observations that are close to
the cut-off (above and below). Let's run the unconstrained model from
\textbf{Exercise 7}, but this time we only use observations that are
within 0.5 percentage points above and below the threshold.

\textbf{Exercise 9: Run the same model used in \texttt{Exercise\ 7}, but
subset your data taking only observations that are above and below 0.5
points from the threshold. Store the results from this regression into
an object and call it \texttt{local}. Use the \texttt{summary()} to
check your results. If you don't remember how to subset data in the
\texttt{lm()} function. See the syntax below}

\begin{Shaded}
\begin{Highlighting}[]
\FunctionTok{lm}\NormalTok{(outcome }\SpecialCharTok{\textasciitilde{}}\NormalTok{ variable1 }\SpecialCharTok{+}\NormalTok{ variable2 }\SpecialCharTok{+}\NormalTok{ variable1 }\SpecialCharTok{*}\NormalTok{ variable2, }\AttributeTok{data=}\NormalTok{data,}
           \AttributeTok{subset=}\NormalTok{(running\_variable}\SpecialCharTok{\textgreater{}={-}}\FloatTok{0.5} \SpecialCharTok{\&}\NormalTok{ running\_variable}\SpecialCharTok{\textless{}=}\FloatTok{0.5}\NormalTok{))}
\end{Highlighting}
\end{Shaded}

\hfill\break

\emph{Reveal Answer}

\begin{Shaded}
\begin{Highlighting}[]
\NormalTok{local }\OtherTok{\textless{}{-}} \FunctionTok{lm}\NormalTok{(school\_women }\SpecialCharTok{\textasciitilde{}}\NormalTok{ margin }\SpecialCharTok{+}\NormalTok{ treat }\SpecialCharTok{+}\NormalTok{ treat }\SpecialCharTok{*}\NormalTok{ margin, }\AttributeTok{data=}\NormalTok{educ,}
           \AttributeTok{subset=}\NormalTok{(margin}\SpecialCharTok{\textgreater{}={-}}\FloatTok{0.5} \SpecialCharTok{\&}\NormalTok{ margin}\SpecialCharTok{\textless{}=}\FloatTok{0.5}\NormalTok{))}
\FunctionTok{summary}\NormalTok{(local)}
\end{Highlighting}
\end{Shaded}

We can see that using a local polynomial function, we find that the
effect of Islamic rule is inconclusive in this case.

\hfill\break

It is important to stress that modern empirical work using RDDs
empirical work employs local polynomial methods. In this case, we are
estimating the average outcomes for treated and untreated units using
observations that are near the cut-off. This approach tends to be more
robust and less sensitive to boundary and overfitting problems. In local
polynomial point estimation, we are still using linear regression, but
within a specific bandwidth near the threshold. In the following
section, we will look at how to use the approach using a non-parametric
estimation strategy.

\begin{center}\rule{0.5\linewidth}{0.5pt}\end{center}

\hypertarget{non-parametric-estimation}{%
\section{Non-Parametric Estimation}\label{non-parametric-estimation}}

Let's now estimate the LATE using a non-parametric estimator.
Conveniently, we can easily do so by using the \texttt{rdrobust}
package. As the name indicates, the package allows us to estimate robust
measurements of uncertainty such as standard errors and confidence
intervals. It is based on theoretical and technical work by Calonico,
Cattaneo and Titiunik. \texttt{rdrobust} estimates robust bias-corrected
confidence intervals that address the problem of undersmoothing
conventional confidence intervals face in RDDs. In other words, a small
bias would be required for them to be valid, which might not be the
case. Moreover, they also address the poor performance of (non-robust)
bias-corrected confidence intervals.

As suggested by the authors and somewhat counter-intuitively, we
therefore use the point estimate provided by the conventional estimator,
but robust standard errors, confidence intervals and p-values to make
statistical inferences.

The \texttt{rdrobust} command has the following minimal syntax. You can
use a uniform bandwidth or specify two different ones. We will work with
further arguments later. Note that you do not need to specify if the
running variable is centred on the cut-off as you can manually specify
the cut-off using the \texttt{c}-argument.

\begin{Shaded}
\begin{Highlighting}[]
\NormalTok{robust\_model }\OtherTok{=} \FunctionTok{rdrobust}\NormalTok{(data}\SpecialCharTok{$}\NormalTok{running\_var, data}\SpecialCharTok{$}\NormalTok{dependent\_var, }
                       \AttributeTok{c=}\NormalTok{[cutoff], }\AttributeTok{h=}\NormalTok{[bandwidth])}
\end{Highlighting}
\end{Shaded}

\textbf{Exercise 10: Estimate the LATE using \texttt{rdrobust} with a
bandwidth of 5\% on either side of the cutoff. Interpret your result.}\\

\emph{Reveal Answer}

\begin{Shaded}
\begin{Highlighting}[]
\NormalTok{robust\_5}\OtherTok{=}\FunctionTok{rdrobust}\NormalTok{(educ}\SpecialCharTok{$}\NormalTok{school\_women, educ}\SpecialCharTok{$}\NormalTok{margin, }\AttributeTok{c=}\DecValTok{0}\NormalTok{, }\AttributeTok{h=}\FloatTok{0.05}\NormalTok{)}
\FunctionTok{summary}\NormalTok{(robust\_5)}
\end{Highlighting}
\end{Shaded}

Our point estimate is \texttt{0.023}. That is, a victory of an Islamic
party would be associated with an increase in the rate of women who
complete school by \texttt{2.3\%} - however, the p-value and confidence
intervals indicate that the estimate is not statistically significant.
Therefore, based on this model, we would conclude that winning an
election does \emph{not} have a an effect on women schooling.

\hfill\break

A bandwidth of 5\% seems about reasonable. But we should better check
different ones, too. Let's see what happens if we halve the bandwidth.

\textbf{Exercise 11: Estimate the same model as before with a bandwidth
of 2.5\%. Report and interpret your results.}\\

\emph{Reveal Answer}

\begin{Shaded}
\begin{Highlighting}[]
\NormalTok{robust\_25}\OtherTok{=}\FunctionTok{rdrobust}\NormalTok{(educ}\SpecialCharTok{$}\NormalTok{school\_women, educ}\SpecialCharTok{$}\NormalTok{margin, }\AttributeTok{c=}\DecValTok{0}\NormalTok{, }\AttributeTok{h=}\FloatTok{0.025}\NormalTok{)}
\FunctionTok{summary}\NormalTok{(robust\_25)}
\end{Highlighting}
\end{Shaded}

Well, we can see that the number of observations used to estimate the
effect has been reduced - which is reasonable and we should be aware of.
The point estimate did not change much and, as before, we find that
there is in fact no significant effect at the cutoff.

\hfill\break

Bandwidths of 5\% or 2.5\% around the cutoff seem somewhat reasonable in
this case - but so would several others. How can we know what bandwidth
we should use to estimate our effect?

Recall the trade-off we are facing when choosing bandwidths that was
discussed in the lecture: On the one hand we know that more narrow
bandwidths are associated with less biased estimates - we rely on units
that are indeed comparable: their distance in the running variable being
as-if random the closer we get to the cut-off. On the other hand: the
wider the bandwidths, the smaller the variance. As in several cases
before, the structure of our data, such as the number of observations,
plays an important role. Even if small bandwidths are desirable, it can
be hard {[}impossible{]} to estimate a robust and significant effect if
the number of observations around the cut-off is very small - even if
there is a \emph{true} effect.

Luckily, the \texttt{rdrobust} package provides a remedy for this. The
packages allows us to specify that we want to use bandwidths that are
optimal given the data input. The \texttt{rdrobust} command then picks
the bandwidth that optimises the \emph{mean square error} - in other
words, \emph{MSE-optimal} bandwidths. Note that this is the default
bandwidth if you don't specify any. Let's try to find out what this
would be in our case.

\textbf{Exercise 12: Estimate the LATE using \texttt{rdrobust} and
MSE-optimal bandwidths. To specify the model, replace the \texttt{h}
argument with \texttt{bwselect="mserd"}.}\\

\emph{Reveal Answer}

\begin{Shaded}
\begin{Highlighting}[]
\NormalTok{robust\_mserd}\OtherTok{=}\FunctionTok{rdrobust}\NormalTok{(educ}\SpecialCharTok{$}\NormalTok{school\_women, educ}\SpecialCharTok{$}\NormalTok{margin, }\AttributeTok{c=}\DecValTok{0}\NormalTok{, }\AttributeTok{bwselect=}\StringTok{"mserd"}\NormalTok{)}
\FunctionTok{summary}\NormalTok{(robust\_mserd)}
\end{Highlighting}
\end{Shaded}

This now looks very different. We estimate a LATE of about \texttt{3\%},
which is significant at the 90\% lvel. Note that the optimal bandwidths
has been estimated to be \texttt{17.2\%-points} on either side of the
cut-off, with a separate optimal bandwidth for bias correction. Note
that the \emph{MSE-optimal} bandwidth is optimal in statistical terms -
we should always make sure to asses the bandwidths against our theory.
Here, we'd compare parties' winning margins up to \texttt{17\%-points}.

\hfill\break

Note that, so far, we have used a single bandwidth for data below and
above the cut-off. We can also specify different ones - both manually
and in terms of optimal bandwidths. As the structure of our data might
differ, different bandwidths might be optimal. We can specify two
different \emph{MSE-optimal} bandwidth selectors by specifying
\texttt{bwselect="msetwo"}.

\textbf{Exercise 13: Estimate the LATE using \texttt{rdrobust} and two
MSE-optimal bandwidths. Interpret your results and compare it the model
with a single optimal bandwidth.}

\emph{Reveal Answer}

\begin{Shaded}
\begin{Highlighting}[]
\NormalTok{robust\_msetwo}\OtherTok{=}\FunctionTok{rdrobust}\NormalTok{(educ}\SpecialCharTok{$}\NormalTok{school\_women, educ}\SpecialCharTok{$}\NormalTok{margin, }\AttributeTok{c=}\DecValTok{0}\NormalTok{, }\AttributeTok{bwselect=}\StringTok{"msetwo"}\NormalTok{)}
\FunctionTok{summary}\NormalTok{(robust\_msetwo)}
\end{Highlighting}
\end{Shaded}

This now looks pretty similar to the single optimal bandwidth, which is
a good sign. In fact, the optimal bandwidth for data points above the
cut-off remains virtually unchanged. For the ones below the cut-off, the
bandwidths is slightly extended. The point estimate and measures of
uncertainty also remain virtually unchanged. If specifying two optimal
bandwidths alters the results significantly, this is an indicator that
the data should be inspected closely for the cause of the diverging
bandwidths.

\hfill\break

\begin{center}\rule{0.5\linewidth}{0.5pt}\end{center}

\hypertarget{kernels}{%
\subsection{Kernels}\label{kernels}}

One of the advantages of the non-parametric RDD is that observations can
be weighted based on their proximity to the cutoff. This is based on the
idea that those closer to the cut-off provide better comparisons than
those further away. Think of victory margins as a running variable: If
we look at fractions of \%-points around the cut-off, it is fair to say
that victory is \emph{as-if random}. The farther away we move, the less
plausible this statement becomes.

As you learned in the lecture, there are different ways to do so. The
default kernel used by the \texttt{rdrobust} package is a
\emph{triangular} kernel - which continuously assigns higher weight to
observations closer to the cut-off. This is the one we have been using
so far as we didn't explicitly specify the kernel. As the figure below
shows, other possible options include \emph{epachnechnikov} and
\emph{uniform} kernels. These can be specified via the \texttt{kernel}
argument.\\

\includegraphics{kernel.JPG}\\

\textbf{Exercise 14: Estimate the LATE using \texttt{rdrobust} and a
single MSE-optimal bandwidths but specifying a uniform kernel. Interpret
your results.}

\emph{Reveal Answer}

\begin{Shaded}
\begin{Highlighting}[]
\NormalTok{robust\_uniform}\OtherTok{=}\FunctionTok{rdrobust}\NormalTok{(educ}\SpecialCharTok{$}\NormalTok{school\_women, educ}\SpecialCharTok{$}\NormalTok{margin, }\AttributeTok{c=}\DecValTok{0}\NormalTok{, }\AttributeTok{bwselect=}\StringTok{"mserd"}\NormalTok{, }\AttributeTok{kernel=}\StringTok{"uniform"}\NormalTok{)}
\FunctionTok{summary}\NormalTok{(robust\_uniform)}
\end{Highlighting}
\end{Shaded}

As compared to the \emph{triangular} kernel, our point estimate is
slightly larger and we find that our robust measures of uncertainty show
that the effect is significant at the 95\% level. Note that the optimal
bandwidth is slightly narrower in the model with a \emph{uniform}
kernel. We have to be aware that this model is less conservative though:
We assign the same weight to each observation, independent of its
distance to the cutoff - which introduces growing bias over increasing
bandwidths.

\hfill\break

\begin{center}\rule{0.5\linewidth}{0.5pt}\end{center}

\hypertarget{falsification-checks}{%
\section{Falsification Checks}\label{falsification-checks}}

\hypertarget{sensitivity}{%
\subsection{Sensitivity}\label{sensitivity}}

If you apply an RDD, you should always make sure that results are robust
to different specifications of the model. Importantly, this involves
\emph{sensitivity checks}. This means you should make sure that results
are robust across model specifications and, importantly, different
bandwidths.

While there are packages that help produce useful plots that visualise
robustness across bandwidths, they can be cumbersome to adjust and are
not very flexible. Hence, one avenue to create such a plot can be to
loop \texttt{rdrobust} models over different bandwidths. This solution
involves somewhat more coding, but gives you a lot of flexibility as to
what you want to display and highlight.

\hfill\break

\textbf{Exercise 15: Create a plot that shows estimates and confidence
intervals for estimates of the effect across different bandwidths.}

There are several ways to do this. Feel free to play around a bit and
try to come up with your own way.

\emph{Reveal Hint}

Hint: You could use the following approach:\\
- Create a data frame with all variables you need for the plot and a and
observation for each bandwidth.\\
- Extract the values from the \texttt{rdrobust} output which you need
for your plot.\\
- Loop the regression and the extraction of output over the bandwidths
indicated in the initial data frame.\\
- Save the output in your loop to the respective row in the data
frame.\\
- Plot the output from the newly created data frame.

\hfill\break

\emph{Reveal Answer}

\begin{Shaded}
\begin{Highlighting}[]
\NormalTok{bandwidth }\OtherTok{\textless{}{-}}  \FunctionTok{seq}\NormalTok{(}\AttributeTok{from =} \FloatTok{0.05}\NormalTok{, }\AttributeTok{to =} \DecValTok{1}\NormalTok{, }\AttributeTok{by =} \FloatTok{0.05}\NormalTok{)  }\CommentTok{\#create a vector with values for each bandwidth you want to estimate}

\NormalTok{coefficient}\OtherTok{\textless{}{-}} \ConstantTok{NA} 
\NormalTok{se }\OtherTok{\textless{}{-}} \ConstantTok{NA}
\NormalTok{obs }\OtherTok{\textless{}{-}} \ConstantTok{NA}
\NormalTok{bw }\OtherTok{\textless{}{-}} \ConstantTok{NA}
\NormalTok{ci\_u }\OtherTok{\textless{}{-}} \ConstantTok{NA}
\NormalTok{ci\_l }\OtherTok{\textless{}{-}} \ConstantTok{NA}

\NormalTok{data\_extract }\OtherTok{\textless{}{-}} \FunctionTok{data.frame}\NormalTok{(bandwidth, coefficient, se, obs, bw, ci\_u, ci\_l) }\CommentTok{\# create a data.frame with all variables you want to incldue in your dataset}

\CommentTok{\# create a loop for each bandwidth that is indicated by \textquotesingle{}i\textquotesingle{}}
\ControlFlowTok{for}\NormalTok{(i }\ControlFlowTok{in}\NormalTok{ bandwidth)\{}
\NormalTok{ rdbw }\OtherTok{\textless{}{-}} \FunctionTok{rdrobust}\NormalTok{(educ}\SpecialCharTok{$}\NormalTok{school\_women, educ}\SpecialCharTok{$}\NormalTok{margin, }\AttributeTok{c=}\DecValTok{0}\NormalTok{, }\AttributeTok{h=}\NormalTok{i) }\CommentTok{\# run the model}
                                      
\CommentTok{\# extract the model output (make sure to extract *robust* statistics)}
\NormalTok{data\_extract}\SpecialCharTok{$}\NormalTok{coefficient[data\_extract}\SpecialCharTok{$}\NormalTok{bandwidth}\SpecialCharTok{==}\NormalTok{i]  }\OtherTok{\textless{}{-}}\NormalTok{ rdbw}\SpecialCharTok{$}\NormalTok{coef[}\DecValTok{3}\NormalTok{] }
\NormalTok{data\_extract}\SpecialCharTok{$}\NormalTok{se[data\_extract}\SpecialCharTok{$}\NormalTok{bandwidth}\SpecialCharTok{==}\NormalTok{i]  }\OtherTok{\textless{}{-}}\NormalTok{ rdbw}\SpecialCharTok{$}\NormalTok{se[}\DecValTok{3}\NormalTok{]}
\NormalTok{data\_extract}\SpecialCharTok{$}\NormalTok{obs[data\_extract}\SpecialCharTok{$}\NormalTok{bandwidth}\SpecialCharTok{==}\NormalTok{i]  }\OtherTok{\textless{}{-}}\NormalTok{ (rdbw}\SpecialCharTok{$}\NormalTok{N\_h[}\DecValTok{1}\NormalTok{] }\SpecialCharTok{+}\NormalTok{ rdbw}\SpecialCharTok{$}\NormalTok{N\_h[}\DecValTok{2}\NormalTok{]) }
\NormalTok{data\_extract}\SpecialCharTok{$}\NormalTok{bw[data\_extract}\SpecialCharTok{$}\NormalTok{bandwidth}\SpecialCharTok{==}\NormalTok{i]  }\OtherTok{\textless{}{-}}\NormalTok{ (rdbw}\SpecialCharTok{$}\NormalTok{bws[}\DecValTok{1}\NormalTok{, }\DecValTok{1}\NormalTok{]) }
\NormalTok{data\_extract}\SpecialCharTok{$}\NormalTok{ci\_l[data\_extract}\SpecialCharTok{$}\NormalTok{bandwidth}\SpecialCharTok{==}\NormalTok{i]  }\OtherTok{\textless{}{-}}\NormalTok{ rdbw}\SpecialCharTok{$}\NormalTok{ci[}\DecValTok{3}\NormalTok{,}\DecValTok{1}\NormalTok{]}
\NormalTok{data\_extract}\SpecialCharTok{$}\NormalTok{ci\_u[data\_extract}\SpecialCharTok{$}\NormalTok{bandwidth}\SpecialCharTok{==}\NormalTok{i]  }\OtherTok{\textless{}{-}}\NormalTok{ rdbw}\SpecialCharTok{$}\NormalTok{ci[}\DecValTok{3}\NormalTok{,}\DecValTok{2}\NormalTok{]}
\NormalTok{                   \}}

\CommentTok{\# Make sure the coefficient (and all other values) are numeric}
\NormalTok{data\_extract}\SpecialCharTok{$}\NormalTok{coefficient  }\OtherTok{\textless{}{-}} \FunctionTok{as.numeric}\NormalTok{(data\_extract}\SpecialCharTok{$}\NormalTok{coefficient)}

\CommentTok{\# Plot the estimates across bandwidths}
\FunctionTok{ggplot}\NormalTok{(}\AttributeTok{data =}\NormalTok{ data\_extract,}
       \FunctionTok{aes}\NormalTok{(}\AttributeTok{x =}\NormalTok{ bandwidth, }\AttributeTok{y =}\NormalTok{ coefficient)) }\SpecialCharTok{+}
  \FunctionTok{geom\_point}\NormalTok{(}\AttributeTok{size =} \FloatTok{0.8}\NormalTok{) }\SpecialCharTok{+}
  \FunctionTok{geom\_ribbon}\NormalTok{(}\FunctionTok{aes}\NormalTok{(}\AttributeTok{ymin =}\NormalTok{ ci\_l, }\AttributeTok{ymax =}\NormalTok{ ci\_u), }\AttributeTok{alpha =} \FloatTok{0.2}\NormalTok{) }\SpecialCharTok{+}
  \FunctionTok{geom\_hline}\NormalTok{(}\FunctionTok{aes}\NormalTok{(}\AttributeTok{yintercept =} \DecValTok{0}\NormalTok{), }\AttributeTok{col =} \StringTok{"red"}\NormalTok{, }\AttributeTok{linetype =} \DecValTok{2}\NormalTok{) }\SpecialCharTok{+}
  \FunctionTok{coord\_cartesian}\NormalTok{(}\AttributeTok{ylim =} \FunctionTok{c}\NormalTok{(}\SpecialCharTok{{-}}\NormalTok{.}\DecValTok{05}\NormalTok{, }\FloatTok{0.15}\NormalTok{)) }\SpecialCharTok{+}
  \FunctionTok{theme\_minimal}\NormalTok{() }\SpecialCharTok{+}
  \FunctionTok{labs}\NormalTok{(}\AttributeTok{y =} \StringTok{"LATE at Discontinuity"}\NormalTok{, }\AttributeTok{x =} \StringTok{"Bandwidths (Vote Margin)"}\NormalTok{)}
\end{Highlighting}
\end{Shaded}

\hfill\break

\begin{center}\rule{0.5\linewidth}{0.5pt}\end{center}

\hypertarget{sorting}{%
\subsection{Sorting}\label{sorting}}

The key identification assumption of RDDs is that there is no
\emph{sorting} on the running variable. That is, the running variable
must be continuous around the threshold. If this is not the case, we
have a problem: Then there's a good chance that the observations close
to the threshold (or on either side of the threshold) are not random. In
other words: Presence of sorting is usually interpreted as empirical
evidence of self-selection or non-random sorting of units into control
and treatment status. Let's check if sorting is problem here. We can do
this using \emph{density checks}. This means we're looking at the
density of the running variable.\\

There are several ways to check for sorting. Let's start with
\emph{McCrary's density test}. We can use the \texttt{DCdenstiy} command
to conduct both a visual test and statistical test. The syntax is the
following:

\begin{Shaded}
\begin{Highlighting}[]
\FunctionTok{DCdensity}\NormalTok{(running\_var ,}\AttributeTok{cutpoint=}\NormalTok{ [cutoff] ,}\AttributeTok{plot=}\ConstantTok{TRUE}\NormalTok{, }\AttributeTok{ext.out =} \ConstantTok{TRUE}\NormalTok{)}
\FunctionTok{title}\NormalTok{(}\AttributeTok{xlab=}\StringTok{"X Lab"}\NormalTok{,}\AttributeTok{ylab=}\StringTok{"Y Lab"}\NormalTok{)}
\FunctionTok{abline}\NormalTok{(}\AttributeTok{v=}\DecValTok{0}\NormalTok{, }\AttributeTok{lty=}\DecValTok{1}\NormalTok{) }\CommentTok{\# Adding a vertical line at the cutoff}
\end{Highlighting}
\end{Shaded}

\textbf{Exercise 16: Examine the density of the running variable using
the \texttt{DCdensity} command.}

\emph{Reveal Answer}

\begin{Shaded}
\begin{Highlighting}[]
\FunctionTok{DCdensity}\NormalTok{(educ}\SpecialCharTok{$}\NormalTok{margin ,}\AttributeTok{cutpoint=} \DecValTok{0}\NormalTok{ ,}\AttributeTok{plot=}\ConstantTok{TRUE}\NormalTok{, }\AttributeTok{ext.out =} \ConstantTok{TRUE}\NormalTok{)}
\FunctionTok{title}\NormalTok{(}\AttributeTok{xlab=}\StringTok{"Winning Margin"}\NormalTok{,}\AttributeTok{ylab=}\StringTok{"Density"}\NormalTok{) }\CommentTok{\# add labels (always add labels. Really. Always.)}
\FunctionTok{abline}\NormalTok{(}\AttributeTok{v=}\DecValTok{0}\NormalTok{, }\AttributeTok{lty=}\DecValTok{1}\NormalTok{)}
\end{Highlighting}
\end{Shaded}

The test provides relatively clear visual evidence that sorting is not a
problem in this case. There is a clear overlap of the confidence
intervals at the cut-off, so the continuity assumption holds. This
command provides also statistical tests. You can see that information on
the individual cells - or bins - and the number of observations within
them is provided. For now, we focus on the inference test for sorting.
The p-value of about \texttt{0.52} makes clear that we do not reject the
null hypothesis of continuity at the cut-off.

\hfill\break

Let's further explore the structure of our data and check if sorting
really isn't a problem here. Recall, we want to make sure the density of
units should be continuous near the cut-off. We can use the
\texttt{rddensity} package to do so. This package applies a different
test than the one we used before. It also is somewhat more powerful and
provides additional information.

The command is straightforward. You only have to insert the running
variable as argument and, if different from 0, the cutoff. Note that you
can optionally specify bandwidths, as above, using the \texttt{h}
argument. Otherwise, an optimal bandwidth will be specified
automatically. You can then save this as an object and run the
\texttt{summary} command to see the output.

\textbf{Exercise 17: Examine the density of the running variable using
the \texttt{rddensity} command. Interpret your findings. Are they
equivalent to those in the previous exercise?}\\

\emph{Reveal Answer}

\begin{Shaded}
\begin{Highlighting}[]
\NormalTok{rdd }\OtherTok{\textless{}{-}} \FunctionTok{rddensity}\NormalTok{(educ}\SpecialCharTok{$}\NormalTok{margin, }\AttributeTok{c=}\DecValTok{0}\NormalTok{)}
\FunctionTok{summary}\NormalTok{(rdd)}
\end{Highlighting}
\end{Shaded}

This output looks clearer. Note that the test is based on
\emph{quadratic} local polynomial functions. We also get a different
bandwidth - recall that we are solely looking at the density of the
running variable here.

Importantly, the test reveals a p-value of \texttt{0.357}. We again do
not reject the null hypothesis of continuity around the cut-off. Both
tests indicating the same result gives us confidence in doing so.

The second part of the output indicates the results of binomial tests
for the allocation of observations around the cut-off for different
windows. The number of observations is indicated as well as individual
p-values for significance tests, with the null hypothesis being equal
distribution on either side of the cut-off. Note that this can give us
an indication, but modest unbalance - at such disaggregate level - is
not evidence for sorting.

\hfill\break

Let's now plot this density test, too. The \texttt{rdplotdensity}
function allows us to do so easily. It uses the results of the test
conducted in \texttt{rddensity}. The syntax is simple:

\begin{Shaded}
\begin{Highlighting}[]
\FunctionTok{rdplotdensity}\NormalTok{([rdd\_object] ,[running\_var])}
\end{Highlighting}
\end{Shaded}

\hfill\break
We can also add some useful arguments. Feel free to adjust your plot:\\

\begin{longtable}[]{@{}
  >{\raggedright\arraybackslash}p{(\columnwidth - 2\tabcolsep) * \real{0.5600}}
  >{\raggedright\arraybackslash}p{(\columnwidth - 2\tabcolsep) * \real{0.4400}}@{}}
\toprule\noalign{}
\begin{minipage}[b]{\linewidth}\raggedright
Argument
\end{minipage} & \begin{minipage}[b]{\linewidth}\raggedright
Description
\end{minipage} \\
\midrule\noalign{}
\endhead
\bottomrule\noalign{}
\endlastfoot
\texttt{plotRange} & Indicates starting and end point of plot \\
\texttt{plotN} & Number of grid points used on either side of the
cutoff \\
\texttt{CIuniform} & True or False. If true, CI are displayed as
continuous bands \\
\end{longtable}

\hfill\break

\textbf{Exercise 18: Use the \texttt{rplotdensity} to conduct a visual
density check in a data range from \texttt{-0.2} to \texttt{0.2}. }\\

\emph{Reveal Answer}

\begin{Shaded}
\begin{Highlighting}[]
\NormalTok{plot\_rdd }\OtherTok{\textless{}{-}} \FunctionTok{rdplotdensity}\NormalTok{(rdd,educ}\SpecialCharTok{$}\NormalTok{margin, }\AttributeTok{plotRange =} \FunctionTok{c}\NormalTok{(}\SpecialCharTok{{-}}\NormalTok{.}\DecValTok{2}\NormalTok{, }\FloatTok{0.2}\NormalTok{),  }\AttributeTok{CIuniform =} \ConstantTok{TRUE}\NormalTok{)}
\end{Highlighting}
\end{Shaded}

\hfill\break

Again, we find visual evidence for continuity around the cutoff. Now we
can be confident about sorting not being a problem with this data set.

\hfill\break

\hypertarget{balance}{%
\subsection{Balance}\label{balance}}

One important falsification test is examining whether near the cut-off
treated units are similar to control units in terms of observable
characteristics. The logic of this test is to identify that if units
cannot manipulate the score they receive, there should not be systematic
differences between untreated and treated units with similar values of
the score.

\textbf{Exercise 19: Let's check for balance around the cutoff. Use the
\texttt{rdrobust()} function. Set the cut-off argument \texttt{c} equal
to 0, and the \texttt{bwselect} argument equal to ``mserd''. In this
case, the outcome variables are the covariates that we want to check for
balance. Check if there is balance for the following covariates:
\texttt{log\_pop}, \texttt{school\_men}, \texttt{sex\_ratio},
\texttt{log\_area}. Remember to use the \texttt{summary()}.}

\hfill\break

\emph{Reveal Answer}

\begin{Shaded}
\begin{Highlighting}[]
\FunctionTok{summary}\NormalTok{(}\FunctionTok{rdrobust}\NormalTok{(educ}\SpecialCharTok{$}\NormalTok{log\_pop, educ}\SpecialCharTok{$}\NormalTok{margin, }\AttributeTok{c=}\DecValTok{0}\NormalTok{, }\AttributeTok{bwselect=}\StringTok{"mserd"}\NormalTok{))}
\FunctionTok{summary}\NormalTok{(}\FunctionTok{rdrobust}\NormalTok{(educ}\SpecialCharTok{$}\NormalTok{school\_men, educ}\SpecialCharTok{$}\NormalTok{margin, }\AttributeTok{c=}\DecValTok{0}\NormalTok{, }\AttributeTok{bwselect=}\StringTok{"mserd"}\NormalTok{))}
\FunctionTok{summary}\NormalTok{(}\FunctionTok{rdrobust}\NormalTok{(educ}\SpecialCharTok{$}\NormalTok{sex\_ratio, educ}\SpecialCharTok{$}\NormalTok{margin, }\AttributeTok{c=}\DecValTok{0}\NormalTok{, }\AttributeTok{bwselect=}\StringTok{"mserd"}\NormalTok{))}
\FunctionTok{summary}\NormalTok{(}\FunctionTok{rdrobust}\NormalTok{(educ}\SpecialCharTok{$}\NormalTok{log\_area, educ}\SpecialCharTok{$}\NormalTok{margin, }\AttributeTok{c=}\DecValTok{0}\NormalTok{, }\AttributeTok{bwselect=}\StringTok{"mserd"}\NormalTok{))}
\end{Highlighting}
\end{Shaded}

We observe that there is balance across all the covariates of interest.

\hfill\break

\hypertarget{placebo-cut-offs}{%
\subsection{Placebo cut-offs}\label{placebo-cut-offs}}

Another falsification test is to identify treatment effects at
artificial or placebo cut-off values. Recall from the lecture that one
of the identifying assumptions is the continuity (or lack of abrupt
changes) of the regression functions for the treated and control units
around the cut-off. While we cannot empirically test this assumption, we
can test continuity apart from the threshold. Although this does not
imply the continuity assumption hold at the cut-off, it helps to rule
out discontinuities other than the cut-off.

This test implies looking at the outcome variable but using different
cut-offs where we should not expect changes in the outcome. Thus, the
estimates from this test should be near zero (and not statistically
significant). One \textbf{important} step to conduct this test is that
you need to split the sample for those observations that are above the
cut-off and those that are below. We do this to avoid ``contamination''
due to the real treatment effects. It also ensures that the analysis of
each placebo cut-off is conducted using only observations with the same
treatment status.

\textbf{Exercise 20: Conduct placebo cut-off test using the following
placebo values: -0.2 and -0.25 for the placebo tests below the cut-off.
Conduct the same test above the cut-off using the following placebo
values: 0.20, and 0.25. Use the \texttt{rdrobust()} function to perform
this test. Set the argument \texttt{bwselect} equal to ``mserd''.
Replace the cut-off argument \texttt{c} with the values listed before.
Use the \texttt{summary()} to report the results. Remember to create two
new data frames. You can use the filter function from tidyverse to do
this.}

\begin{longtable}[]{@{}
  >{\raggedright\arraybackslash}p{(\columnwidth - 2\tabcolsep) * \real{0.5600}}
  >{\raggedright\arraybackslash}p{(\columnwidth - 2\tabcolsep) * \real{0.4400}}@{}}
\toprule\noalign{}
\begin{minipage}[b]{\linewidth}\raggedright
Function/argument
\end{minipage} & \begin{minipage}[b]{\linewidth}\raggedright
Description
\end{minipage} \\
\midrule\noalign{}
\endhead
\bottomrule\noalign{}
\endlastfoot
\texttt{new\_data\ \textless{}-\ data} & Creates a new data frame using
the conditions set before the pipeline operator
\texttt{\%\textgreater{}\textgreater{}\%} \\
\texttt{filter(variable\ ==\ "condition")} & Subset the data based on a
logical operation \\
\end{longtable}

\begin{Shaded}
\begin{Highlighting}[]
\NormalTok{new\_data }\OtherTok{\textless{}{-}}\NormalTok{ data }\SpecialCharTok{\%\textgreater{}\%} 
  \FunctionTok{filter}\NormalTok{(variable }\SpecialCharTok{\textgreater{}=} \DecValTok{0}\NormalTok{)}
\end{Highlighting}
\end{Shaded}

\hfill\break

\emph{Reveal Answer}

\begin{Shaded}
\begin{Highlighting}[]
\NormalTok{educ\_above }\OtherTok{\textless{}{-}}\NormalTok{ educ }\SpecialCharTok{\%\textgreater{}\%} 
  \FunctionTok{filter}\NormalTok{(margin }\SpecialCharTok{\textgreater{}=} \DecValTok{0}\NormalTok{)}

\NormalTok{educ\_below }\OtherTok{\textless{}{-}}\NormalTok{ educ }\SpecialCharTok{\%\textgreater{}\%} 
  \FunctionTok{filter}\NormalTok{(margin }\SpecialCharTok{\textless{}} \DecValTok{0}\NormalTok{)}

\FunctionTok{summary}\NormalTok{(}\FunctionTok{rdrobust}\NormalTok{(educ\_above}\SpecialCharTok{$}\NormalTok{school\_women, educ\_above}\SpecialCharTok{$}\NormalTok{margin, }\AttributeTok{c=}\FloatTok{0.2}\NormalTok{, }\AttributeTok{bwselect =} \StringTok{"mserd"}\NormalTok{)) }\CommentTok{\# {-}0.044}
\FunctionTok{summary}\NormalTok{(}\FunctionTok{rdrobust}\NormalTok{(educ\_above}\SpecialCharTok{$}\NormalTok{school\_women, educ\_above}\SpecialCharTok{$}\NormalTok{margin, }\AttributeTok{c=}\FloatTok{0.25}\NormalTok{, }\AttributeTok{bwselect=}\StringTok{"mserd"}\NormalTok{)) }\CommentTok{\#{-}0.025}

\FunctionTok{summary}\NormalTok{(}\FunctionTok{rdrobust}\NormalTok{(educ\_below}\SpecialCharTok{$}\NormalTok{school\_women, educ\_below}\SpecialCharTok{$}\NormalTok{margin, }\AttributeTok{c=}\SpecialCharTok{{-}}\FloatTok{0.2}\NormalTok{, }\AttributeTok{bwselect=}\StringTok{"mserd"}\NormalTok{)) }\CommentTok{\# 0.063}
\FunctionTok{summary}\NormalTok{(}\FunctionTok{rdrobust}\NormalTok{(educ\_below}\SpecialCharTok{$}\NormalTok{school\_women, educ\_below}\SpecialCharTok{$}\NormalTok{margin, }\AttributeTok{c=}\SpecialCharTok{{-}}\FloatTok{0.25}\NormalTok{, }\AttributeTok{bwselect=}\StringTok{"mserd"}\NormalTok{)) }\CommentTok{\# {-}0.008}
\end{Highlighting}
\end{Shaded}

We find that for almost all the placebo cut-off values, the coefficients
are all nearly zero and not statistically significant (with the
exception of -0.2, which we would want to investigate further). This
evidence suggests that the assumption of continuity is likely to hold in
this case.

\hfill\break



\end{document}
